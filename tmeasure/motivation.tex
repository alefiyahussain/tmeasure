\section{Motivation}
\label{sec:motive}

There are more than 50 emulab-based testbeds across the world~\cite{EmulabWorld}. 
For example,  DETERLab is one example of an emulab-based testbed located 
 at USC Information Sciences Institute with 566 general computers and 
 specialized hardware computers. 
In this paper, we will use the term Emulab to refer to all 
 such testbeds.  
 
% what are we trying to do  
Emulab-based testbeds are widely used for networking and 
distributed system research. 
These different emulab-based testbeds 
%use variations 
% of the software system to 
 use a common ns-2-based experiment description for 
  creating different topological structures 
 and configuring the experiment nodes. 
This has 
 enabled networking researchers to rapidly port experiment descriptions 
  from one testbed to another. 
However, we observed very different results when 
 using the \emph{same} experiment description 
 on different testbeds. 
 
We believe these difference in results are due to variations in both the 
 facility hardware and the software system implementations across different 
 Emulab-based testbeds. 
These variations impact the measurements taken on the testbed. 
Researchers can rarely account for them, they are typically not adjusted for. 
However, they impact the repeatability and the correctness of the results. 
 
\subsection{Background} 
\label{sec:background}

Emulab provides a common interface to 
 configure the resources required for a networked experiment. 
A networked experiment consists of a network topology with 
 one or more end hosts. 
The network topology defines the structure that connects the end hosts with each other. 
A network topology consists of nodes and 
links~\footnote{Emulab also offers creating LANs 
 using similar mechanisms that we do not discuss in the paper.} 

A \emph{node} can be an end host or a router en-route to two end hosts. 
A node's hardware and software can be configured as part of the 
experiment description. 
The node's hardware can be chosen from a 
 predefined set of available hardware types. 
 The node's software, that is the operating systems, libraries and tools, 
  can installed by the researcher. 
A \emph{link} is a connection between two nodes. 
A link's bandwidth,delay and queuing characteristics can be configured 
 as part of the experiment description. 
The link is realized 
 using VLANs and link emulation mechanisms. 
The VLAN creates a virtual point to point link, possibly across multiple switches, 
 on the testbed.
The bandwidth, delay, and queuing characteristics of a link 
 are realized using a delay node that provides the 
 link emulation capabilities. 
Many emulab-based testbeds 
 implement the delay node differently.
For example, the DETERLab uses  
  Click-based delay nodes and does 
 not permit changing queue sizes. 
%They can be usually addresses easily. 
%The experiment description parser parser with correct documentation 
% and parsers within the code to 
 
Each node in emulab is connected to two networks.
An \emph{experiment network} configured based on the network topology 
specified in the experiment description and a \emph{control network} that 
enables access to the nodes to control and configure them. 
In this paper we will focus primarily on the network-level characteristics
 of the experiment and control network and how 
  they impact the experiment results. 

\subsection{Intra-testbed Links} 
Multicast is a IP-level mechanism to provide one to many communication. 
It leverages network-level functionality within the routers to 
  replicate a packet only when the topology of the network requires it. 
%A sender can communicate with a large number of receivers 
 %by sending  one packet and it is replicated by the network so that 
  %all the receivers get a copy of the packet. 
While not used commonly on the Internet, 
 multicast is widely used in emulab-based testbeds for experiment configuration 
 and control.
 Emulab-based testbeds also provide a controlled environment to develop 
 and evaluate multicast applications such as content delivery networks and streaming 
video.  

We observed network-level characteristics of the multicast implementation 
 is different on different testbeds. 
Also, after a recent upgrade of network switches on one of the testbeds, 
 a multicast-based application that was working previously, stopped working. 
Multicast is widely used with UDP-based transport protocols. 
UDP is not reliable and does not guarantee the delivery of a message. 
Hence messages may be dropped, delivered multiple times, or delivered out of order, 
 leading to different performance characteristics as 
 we discuss in Section~\ref{sec:results}


\subsection{Inter-testbed Links}
Many networking experiments include links 
 that span across the Internet. 
In such cases, the link characteristics are governed by the dynamic 
 connectivity and congestion conditions between the two nodes. 
When evaluating a distributed system across 
 such a link, it is important to 
correctly characterize and account of the dynamics in the 
 network link properties. 
 
For example, when exploring energy cyber physical models
 based on dependent, time synchronized
CPS models spanning the across testbeds.
Such models are delay sensitive and require hard 
 real time guarantees. Hence it is important
to quantify the delay properties of the underlining cyber
substrate so that the models can correctly designed for
this environment.



We need to have a mechanism that  
 ensure the network-level constraints are exposed 
  to the design of the experiment. 
The network-level assumption, about delay, bw, are invariant 
 and hold throughout the duration of the 
experiment. 
Hence to expose these underlying network-level 
  constraints and need to provide mechanisms that allow the 
  experimenter to ensure the required invariants hold 
  through put the experiment    

% Hard
Testbed environments are complex. 
Virtualization is prevalent, federation is common 
 mininet and simulation environments combined 
  with emulation for scale. 
 Ensuring the correctness of the environment 
  that is, making the sure the requested 
  network-level characteristics are satisfied is not easy. 
  


     
 
 


